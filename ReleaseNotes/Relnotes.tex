\chapterimage{TOC.jpg} % Chapter heading image
\chapter*{Release history}

\begin{itemize}
\item[] \textbf{2019, November : Release version 2.0}\newline
    \begin{itemize}
    \item[] \textbf{New features}
        \begin{itemize}
            \item[] Support for Java 11
            \item[] Federation of dataClay instances
            \item[] Support for ARM 32-bit architecture
            \item[] Configurable reverse proxy in communications between dataClay instances
            \item[] Secure TLS client-server communications
            \item[] Mixins in Python
            \item[] Support to AspectJ
            \item[] Configurable tracing
        \end{itemize}
    \item[] \textbf{Improvements}
        \begin{itemize}
            \item[] Bootstrap performance
            \item[] Performance in inter-service communications
            \item[] Management of in-memory objects in memory intensive Python applications
            \item[] Metadata caching mechanisms in the Python Execution Environment
            \item[] Docker infrastructure and Docker images: health check, minimization of ports exposed, reduced container size 
            \item[] Easier deployment
            \item[] New deployable demos and updated examples
            \item[] Bug fixes
        \end{itemize}
    \item[] \textbf{Deprecation notice:}
        \begin{itemize}
            \item[] dataClay 2.0 source code is the last release supporting Python 2. Deployment and services are provided for Python 3.
        \end{itemize}
    \end{itemize}
\end{itemize}
