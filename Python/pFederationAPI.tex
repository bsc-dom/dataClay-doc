\section{Object extended methods: federation}
\label{sec:pObjectFederation}

There are cases where it is interesting to be able to access an object stored in a dataClay instance from another dataClay instance as if it belonged to both instances. More details on federation is presented in Section~\ref{sec:pFederation}.


\begin{tBox}
\textcolor{basecolor} {\bf Example: Using federate}
\begin{python}
from dataclay import api
p1 = Person.get_by_alias("person1")
# federating object and subobjects, from current dataClay to dataClay2
p1.federate("dataClay2")
\end{python}
\end{tBox}

% ------- getFederation ---------

\begin{dBox}
\texttt{def \CALL{get\_federation\_of\_object}(self):}
\LINE

{\it Description:}

\begin{itemize}
 \item Retrieves all the dataClay instances where the object is federated. 
 A DataClayInstance has a the id, name, hostname and port of a dataClay instance.
\end{itemize}

{\it Returns:}

\begin{itemize}
 \item A set of DataClayInstance objects in which this object is federated. 
 It can be empty if it is not federated.
\end{itemize}

\end{dBox}

\begin{tBox}
\textcolor{basecolor} {\bf Example: Using get\_federation\_of\_object}
\begin{python}
from dataclay import api
newPerson = Person.get_by_alias('Alias')
dataclays = list(p1.get_federation_of_object())
assert api.LOCAL in dataclays
\end{python}
\end{tBox}

% ------- federate ---------

\begin{dBox}
\texttt{def \CALL{federate}(self, dataclay\_name=None, recursive=True):}
\LINE

{\it Description:}

\begin{itemize}
	\item Federates current object with external dataClay. For more information about federation, check section \ref{sec:pFederation}.
\end{itemize}

{\it Parameters:}

\begin{itemize}
  \item \texttt{\bfseries dataclay\_name:} Name of the external dataClay. It must be previously registered.
  \item \texttt{\bfseries recursive:} when this flag is TRUE, all objects referenced by the current one will also be federated (except those that are already present in the destination dataClay). When this parameter is not set, the default behavior is to perform a recursive federation.
\end{itemize}
{\it Returns:}

{\it Exceptions:}

\begin{itemize}
  \item If the object is not persistent or the external dataClay is not registered, a DataClayException is raised.

\end{itemize}

\end{dBox}

% ------- runFederated ---------

\begin{dBox}
\texttt{public Object \CALL{run\_federated}(dc\_info, \newline operation\_name, params) throws DataClayException}
\LINE

{\it Description:}

\begin{itemize}
  \item Executes a specific method on a particular dataClay where the object is federated. Notice that currently this method is intended for synchronization purposes analogously to method \textit{runRemote}.
\end{itemize}

{\it Parameters:}

\begin{itemize}
  \item \texttt{\bfseries dc\_info:} dataClay instance where the method must be executed.
  \item \texttt{\bfseries operation\_name:} ID of the method to be executed.
  \item \texttt{\bfseries params:} The regular parameters of the method.
\end{itemize}
 
{\it Returns:}

\begin{itemize}
  \item The expected result from the execution of the specified method.
\end{itemize}

{\it Exceptions:}

\begin{itemize}
  \item If this object is not federated with given dataClay instance, a DataClayException is raised.
\end{itemize}
\end{dBox}


