In the following we present the API provided by dataClay to manage the federation of objects between dataClay instances. It comprises a set of methods that are part of the dataClay API to manage the connection between different dataClay instances, as well as object methods to manage the federation of objects. Recall that methods from the dataClay API can be called through \textit{dataclay.api} with the proper import, for instance:

\colorbox{basecolor!20}{\texttt{from dataclay.api import finish, init, register\_dataclay}}

\subsection{dataClay API methods}
\label{sec:PythonFederationAPI}

% ------- federateAllObjects ---------

\begin{dBox}
\texttt{def \CALL{federate\_all\_objects}(dataclay\_id):}
\LINE

{\it Description:}

\begin{itemize}
  \item Federates all the objects in the current dataClay instance with another dataClay instance. 
\end{itemize}

{\it Parameters:}

\begin{itemize}
  \item \texttt{\bfseries dataclay\_id:} ID of the external dataClay. It must be previously registered.
\end{itemize}

\end{dBox}

% ------- getDataClayID ---------

\begin{dBox}
\texttt{def \CALL{get\_dataclay\_id}([host, port]):}
\LINE

{\it Description:}

\begin{itemize}
    \item Retrieves the ID of the dataClay instance accessible in \textit{host}, \textit{port}, or of the current dataClay instance if there are no parameters.
\end{itemize}

{\it Parameters:}

\begin{itemize}
  \item \texttt{\bfseries host:} host where the dataClay instance is located.
  \item \texttt{\bfseries port:} port where the dataClay instance is listening.
\end{itemize}

{\it Returns:}

\begin{itemize}
 \item The ID of the current dataClay instance, or of the dataClay instance located in \textit{host}, \textit{port}.
\end{itemize}

\end{dBox}

% ------- registerDataClay ---------

\begin{dBox}
\texttt{def \CALL{register\_dataclay}(host, port):}
\LINE

{\it Description:}

\begin{itemize}
    \item Makes the current dataClay instance aware of another dataClay instance accessible in \textit{host} and \textit{port}, and returns its ID.
\end{itemize}

{\it Parameters:}

\begin{itemize}
  \item \texttt{\bfseries host:} host where the dataClay instance to be registered is located.
  \item \texttt{\bfseries port:} port where the dataClay instance to be registered is listening.
\end{itemize}

{\it Returns:}

\begin{itemize}
 \item The ID of the dataClay instance located in \textit{host}, \textit{port}.
\end{itemize}

\end{dBox}

% ------- unfederateAllObjects ---------
\begin{dBox}

\texttt{def \CALL{unfederate\_all\_objects}([dc\_id]):}
\LINE

{\it Description:}

\begin{itemize}
  \item Unfederates all the objects in the current dataClay instance with the indicated dataClay instance. If no \textit{dc\_id} is specified, the objects are unfederated from all the instances where they live.
\end{itemize}

{\it Parameters:}

\begin{itemize}
  \item \texttt{\bfseries dc\_id:} ID of the external dataClay. It must be previously registered.
\end{itemize}

\end{dBox}

\subsection{Object methods}
\label{sec:PythonFederationObject}

% ------- federate ---------
\begin{dBox}
\texttt{def \CALL{federate}(self, dc\_id, recursive=True):}
\LINE

{\it Description:}

\begin{itemize}
  \item Federates current object with another dataClay instance. 
\end{itemize}

{\it Parameters:}

\begin{itemize}
  \item \texttt{\bfseries dc\_id:} ID of the external dataClay. It must be previously registered.
  \item \texttt{\bfseries recursive:} when this flag is TRUE, all objects (recursively) referenced by the current one will also be federated (except those that are already present in the destination dataClay). 
\end{itemize}
\end{dBox}


\begin{tBox}
\textcolor{basecolor} {\bf Example: Using federate}
\begin{python}
  other_dc = get_dataclay_id(host, port);
  p1 = Person.get_by_alias("person1");
  # federating object and subobjects to other_dc (previously registered)
  p1.federate(other_dc);
\end{python}
\end{tBox}

% ------- getFederationSource ------

\begin{dBox}
\texttt{def \CALL{get\_federation\_source}(self):}
\LINE

{\it Description:}

\begin{itemize}
 \item Retrieves the ID of the dataClay instance where the object is federated from. 
\end{itemize}

{\it Returns:}

\begin{itemize}
 \item The id of the dataClay instance that is the source of this federated object.  
 It is null if the object is not federated.
\end{itemize}

\end{dBox}

% ------- getFederationTargets ---------

\begin{dBox}
\texttt{def \CALL{get\_federation\_targets}(self):}
\LINE

{\it Description:}

\begin{itemize}
 \item Retrieves the IDs of all the dataClay instances where the object is federated. 
\end{itemize}

{\it Returns:}

\begin{itemize}
 \item A set of DataClayInstanceID objects in which this object is federated. 
 It can be empty if it is not federated.
\end{itemize}

\end{dBox}

\begin{tBox}
\textcolor{basecolor} {\bf Example: Using get\_federation\_targets}
\begin{python}
from dataclay import api
newPerson = Person.get_by_alias('Alias')
dataclays = list(p1.get_federation_of_object())
assert api.LOCAL in dataclays
\end{python}
\end{tBox}

% ------- setInDataClayInstance ---------

\begin{dBox}
\texttt{def \CALL{set\_in\_dataclay\_instance}(self, dc\_id, operation\_name, params):}
\LINE

{\it Description:}

\begin{itemize}
  \item Executes a setter on a particular dataClay where the object is federated.
\end{itemize}

{\it Parameters:}

\begin{itemize}
  \item \texttt{\bfseries dc\_id:} dataClay instance where the method must be executed.
  \item \texttt{\bfseries operation\_name:} ID of the setter to be executed.
  \item \texttt{\bfseries params:} The parameters of the method.
\end{itemize}
 
\end{dBox}

% ------- unfederate ---------
\begin{dBox}

\texttt{def \CALL{unfederate}(self, [dc\_id], recursive=True):}
\LINE

{\it Description:}

\begin{itemize}
  \item Unfederates current object (and referenced objects) with the indicated dataClay instance. If no \textit{dc\_id} is specified, the object is unfederated from all the instances where it lives.
\end{itemize}

{\it Parameters:}

\begin{itemize}
  \item \texttt{\bfseries dc\_id:} ID of the external dataClay. It must be previously registered.
  \item \texttt{\bfseries recursive:} when this flag is TRUE, all objects (recursively) referenced by the current one will also be unfederated.
\end{itemize}

\end{dBox}






