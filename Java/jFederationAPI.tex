\section{Federation mechanism}
\label{sec:JavaObjectFederation}

More details about federation are presented in Section~\ref{sec:jFederation}.

% ------- federate ---------
\begin{dBox}
\texttt{public void \CALL{federate}(DataClayInstanceID dcID [,boolean recursive])}
\LINE

{\it Description:}

\begin{itemize}
  \item Federates current object with another dataClay instance. 
\end{itemize}

{\it Parameters:}

\begin{itemize}
  \item \texttt{\bfseries dcID:} ID of the external dataClay. It must be previously registered.
  \item \texttt{\bfseries recursive:} when this flag is TRUE, all objects (recursively) referenced by the current one will also be federated (except those that are already present in the destination dataClay). This parameter is optional, default value is TRUE.
\end{itemize}

\end{dBox}

\begin{tBox}
\textcolor{basecolor} {\bf Example: Using federate}
\begin{java}
DataClayID otherDC = DataClay.getDataClayID(host, port);
Person p1 = Person.getByAlias("person1");
// federating object and subobjects to otherDC (previously registered)
p1.federate(otherDC);
\end{java}
\end{tBox}

% ------- getFederationSource ------

\begin{dBox}
\texttt{public DataClayInstanceID \CALL{getFederationSource}()}
\LINE

{\it Description:}

\begin{itemize}
 \item Retrieves the ID of the dataClay instance where the object is federated from. 
\end{itemize}

{\it Returns:}

\begin{itemize}
 \item A DataClayInstanceID which is the source of this federated object.  
 It is null if the object is not federated.
\end{itemize}

\end{dBox}

% ------- getFederationTargets ------

\begin{dBox}
\texttt{public Set<DataClayInstanceID> \CALL{getFederationTargets}()}
\LINE

{\it Description:}

\begin{itemize}
 \item Retrieves the IDs of all the dataClay instances where the object is federated to. 
\end{itemize}

{\it Returns:}

\begin{itemize}
 \item A set of DataClayInstanceID objects in which this object is federated. 
 It is empty if the object is not federated.
\end{itemize}

\end{dBox}

\begin{tBox}
\textcolor{basecolor} {\bf Example: Using getFederationTargets}
\begin{java}
Person p1 = Person.getByAlias("personalias");
// using getFederationTargets to check if p1 is federated
Set<DataClayInstanceID> federation = p1.getFederationTargets();
return (!federation.size() == 0)
\end{java}
\end{tBox}

% ------- synchronizeFederated --------

\begin{dBox}
\texttt{public void \CALL{synchronizeFederated}(DataClayInstanceID dcID, \newline ImplementationID implID, Object[] params)} 
\LINE

{\it Description:}

\begin{itemize}
  \item Executes an implementation on a particular dataClay where the object is federated, for synchronization purposes.
\end{itemize}

{\it Parameters:}

\begin{itemize}
  \item \texttt{\bfseries dcID:} dataClay instance where the method must be executed.
  \item \texttt{\bfseries pimlID:} ID of the implementation to be executed.
  \item \texttt{\bfseries params:} The parameters of the method.
\end{itemize}
 
\end{dBox}

% ------- unfederate ---------
\begin{dBox}

\texttt{public void \CALL{unfederate}([DataClayInstanceID dcID] [,boolean recursive])}
\LINE

{\it Description:}

\begin{itemize}
  \item Unfederates current object (and referenced objects) with the indicated dataClay instance. If no dataClayID is specified, the object is unfederated from all the instances where it lives.
\end{itemize}

{\it Parameters:}

\begin{itemize}
  \item \texttt{\bfseries dcID:} ID of the external dataClay. It must be previously registered.
  \item \texttt{\bfseries recursive:} when this flag is TRUE, all objects (recursively) referenced by the current one will also be unfederated. This parameter is optional, default value is TRUE.
\end{itemize}

\end{dBox}

