In the following we present the API provided by dataClay to manage the federation of objects between dataClay instances. It comprises a set of methods that are part of the dataClay API to manage the connection between different dataClay instances, as well as object methods to manage the federation of objects. Recall that methods from the dataClay API can be called through \textit{DataClay} main class by including the import:

\colorbox{basecolor!20}{\texttt{import es.bsc.dataclay.api.DataClay}}

\subsection{dataClay API methods}
\label{sec:JavaFederationAPI}

% ------- federateAllObjects ---------

\begin{dBox}
\texttt{public static DataClayInstanceID \CALL{federateAllObjects}(DataClayInstanceID dcID)}
\LINE

{\it Description:}

\begin{itemize}
  \item Federates all the objects in the current dataClay instance with another dataClay instance. 
\end{itemize}

{\it Parameters:}

\begin{itemize}
  \item \texttt{\bfseries dcID:} ID of the external dataClay. It must be previously registered.
\end{itemize}

\end{dBox}

% ------- getDataClayID ---------

\begin{dBox}
\texttt{public static DataClayInstanceID \CALL{getDataClayID}([String host, String port])}
\LINE

{\it Description:}

\begin{itemize}
    \item Retrieves the ID of the dataClay instance accessible in \textit{host}, \textit{port}, or of the current dataClay instance if there are no parameters.
\end{itemize}

{\it Parameters:}

\begin{itemize}
  \item \texttt{\bfseries host:} host where the dataClay instance is located.
  \item \texttt{\bfseries port:} port where the dataClay instance is listening.
\end{itemize}

{\it Returns:}

\begin{itemize}
 \item The ID of the current dataClay instance, or of the dataClay instance located in \textit{host}, \textit{port}.
\end{itemize}

\end{dBox}

% ------- registerDataClay ---------

\begin{dBox}
\texttt{public static DataClayInstanceID \CALL{registerDataClay}(String host, String port)}
\LINE

{\it Description:}

\begin{itemize}
    \item Makes the current dataClay instance aware of another dataClay instance accessible in \textit{host} and \textit{port}, and returns its ID.
\end{itemize}

{\it Parameters:}

\begin{itemize}
  \item \texttt{\bfseries host:} host where the dataClay instance to be registered is located.
  \item \texttt{\bfseries port:} port where the dataClay instance to be registered is listening.
\end{itemize}

{\it Returns:}

\begin{itemize}
 \item The ID of the dataClay instance located in \textit{host}, \textit{port}.
\end{itemize}

\end{dBox}

% ------- unfederateAllObjects ---------
\begin{dBox}

\texttt{public void \CALL{unfederateAllObjects}([DataClayInstanceID dcID])}
\LINE

{\it Description:}

\begin{itemize}
  \item Unfederates all the objects in the current dataClay instance with the indicated dataClay instance. If no dataClayID is specified, the objects are unfederated from all the instances where they live.
\end{itemize}

{\it Parameters:}

\begin{itemize}
  \item \texttt{\bfseries dcID:} ID of the external dataClay. It must be previously registered.
\end{itemize}

\end{dBox}

\subsection{Object methods}
\label{sec:JavaFederationObject}

% ------- federate ---------
\begin{dBox}
\texttt{public void \CALL{federate}(DataClayInstanceID dcID [,boolean recursive])}
\LINE

{\it Description:}

\begin{itemize}
  \item Federates current object with another dataClay instance. 
\end{itemize}

{\it Parameters:}

\begin{itemize}
  \item \texttt{\bfseries dcID:} ID of the external dataClay. It must be previously registered.
  \item \texttt{\bfseries recursive:} when this flag is TRUE, all objects (recursively) referenced by the current one will also be federated (except those that are already present in the destination dataClay). This parameter is optional, default value is TRUE.
\end{itemize}

\end{dBox}

\begin{tBox}
\textcolor{basecolor} {\bf Example: Using federate}
\begin{java}
DataClayID otherDC = DataClay.getDataClayID(host, port);
Person p1 = Person.getByAlias("person1");
// federating object and subobjects to otherDC (previously registered)
p1.federate(otherDC);
\end{java}
\end{tBox}

% ------- getFederationSource ------

\begin{dBox}
\texttt{public DataClayInstanceID \CALL{getFederationSource}()}
\LINE

{\it Description:}

\begin{itemize}
 \item Retrieves the ID of the dataClay instance where the object is federated from. 
\end{itemize}

{\it Returns:}

\begin{itemize}
 \item A DataClayInstanceID which is the source of this federated object.  
 It is null if the object is not federated.
\end{itemize}

\end{dBox}

% ------- getFederationTargets ------

\begin{dBox}
\texttt{public Set<DataClayInstanceID> \CALL{getFederationTargets}()}
\LINE

{\it Description:}

\begin{itemize}
 \item Retrieves the IDs of all the dataClay instances where the object is federated to. 
\end{itemize}

{\it Returns:}

\begin{itemize}
 \item A set of DataClayInstanceID objects in which this object is federated. 
 It is empty if the object is not federated.
\end{itemize}

\end{dBox}

\begin{tBox}
\textcolor{basecolor} {\bf Example: Using getFederationTargets}
\begin{java}
Person p1 = Person.getByAlias("personalias");
// using getFederationTargets to check if p1 is federated
Set<DataClayInstanceID> federation = p1.getFederationTargets();
return (!federation.size() == 0)
\end{java}
\end{tBox}

% ------- synchronizeFederated --------

\begin{dBox}
\texttt{public void \CALL{synchronizeFederated}(DataClayInstanceID dcID, \newline ImplementationID implID, Object[] params)} 
\LINE

{\it Description:}

\begin{itemize}
  \item Executes an implementation on a particular dataClay where the object is federated, for synchronization purposes.
\end{itemize}

{\it Parameters:}

\begin{itemize}
  \item \texttt{\bfseries dcID:} dataClay instance where the method must be executed.
  \item \texttt{\bfseries pimlID:} ID of the implementation to be executed.
  \item \texttt{\bfseries params:} The parameters of the method.
\end{itemize}
 
\end{dBox}

% ------- unfederate ---------
\begin{dBox}

\texttt{public void \CALL{unfederate}([DataClayInstanceID dcID] [,boolean recursive])}
\LINE

{\it Description:}

\begin{itemize}
  \item Unfederates current object (and referenced objects) with the indicated dataClay instance. If no dataClayID is specified, the object is unfederated from all the instances where it lives.
\end{itemize}

{\it Parameters:}

\begin{itemize}
  \item \texttt{\bfseries dcID:} ID of the external dataClay. It must be previously registered.
  \item \texttt{\bfseries recursive:} when this flag is TRUE, all objects (recursively) referenced by the current one will also be unfederated. This parameter is optional, default value is TRUE.
\end{itemize}

\end{dBox}

