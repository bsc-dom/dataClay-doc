\chapterimage{Coffee.jpg} % Chapter heading image

\chapter{Java API}
\label{sec:JavaAPI}

This chapter presents the Java API that can be used by applications divided into the following sections. First, in Section~\ref{sec:JavaGlobalAPI} we present the API intended to initialize and finish applications, as well as, to gather information about the system. In Section~\ref{sec:JavaObjectStore} we show the API for Object Store operations (GET(CLONE)/PUT/UPDATE). Next, in Section~\ref{sec:JavaObjectExtendedMethods} we introduce extended methods to expand object store operations from an Object-oriented programming perspective. In Section~\ref{sec:JavaObjectAdvanced} we show advanced extensions that will only be needed by a subset of applications. Finally, we present extra concepts such as error handling in Section~\ref{sec:JavaErrorHandling}, replica management in Section~\ref{sec:JavaReplication} and further considerations in Section~\ref{sec:JavaConsiderations}.

Notice that current supported Java version is 1.8 (OpenJDK 8, reference implementation of Java SE 8).

\section{dataClay API}
\label{sec:JavaGlobalAPI}

In this section, we present a set of calls that are not linked to any given object, but are general to the system.

In Java, they can be called through \textit{DataClay} main class by including the import:

\colorbox{basecolor!20}{\texttt{import es.bsc.dataclay.api.DataClay}}


% ------- finish ---------

\begin{dBox}
\texttt{public static void \CALL{finish}() throws DataClayException}
\LINE

{\it Description:}

\begin{itemize}
    \item Finishes a session with dataClay that has been previously created using init.
\end{itemize}

{\it Exceptions:}

\begin{itemize}
    \item If the session is not initialized or an error occurs while finishing the session, a DataClayException is thrown.
\end{itemize}
 
\end{dBox}


% ------- getBackends ---------

\begin{dBox}
\label{call:JavaGetBackends}
\texttt{public static Map<BackendID, Backend> \CALL{getBackends}() }
\LINE

{\it Description:}

\begin{itemize}
    \item Retrieves the available backends in the system.
\end{itemize}
 
{\it Returns:}

\begin{itemize}
    \item A map with the available backends in the system indexed by their backend IDs.
\end{itemize}

{\it Exceptions:}

\begin{itemize}
    \item If the session is not initialized, a DataClayException is thrown.
\end{itemize}

\end{dBox}


% ------- init ---------

\begin{dBox}
\texttt{public static void \CALL{init}() throws DataClayException}
\LINE

{\it Description:}

\begin{itemize}
    \item Creates and initializes a new session with dataClay.
\end{itemize}

{\it Environment:}

\begin{itemize}
    \item \texttt{\bfseries session.properties:} The configuration file can be optionally specified. Location of this file and its contents are detailed in Section~\ref{sec:ClientConfigFiles}.
\end{itemize}

{\it Exceptions:}

\begin{itemize}
    \item If any error occurs while initializing the session, a DataClayException is thrown.
\end{itemize}
 
\end{dBox}

\begin{tBox}
\textcolor{basecolor} {\bf Example: Using dataClay api - distributed people}
\begin{java}
import es.bsc.dataclay.api.DataClay;
import model.Person;

// Open session with init()
DataClay.init();

List<Person> people = new ArrayList<>();
people.add(new Person("Alice", 32));
people.add(new Person("Bob", 41));
people.add(new Person("Charlie", 35));

// Retrieve backend information with getBackends()
Map<BackendID, Backend> backends = DataClay.getBackends();

BackendID[] backendArray = backends.keySet().toArray();
int numBackends = backends.size();
int i = 0;
for (Person p : people) {
  p.dcPut(backendArray[i % numBackends]);
  i++;
}

// Close session with finish()
DataClay.finish();
\end{java}
\end{tBox}


\section{Object store methods}\index{Object Store}
\label{sec:JavaObjectStore}

Object store methods are those related to common \textit{GET(CLONE)/PUT/UPDATE} operations as introduced in Sections~\ref{sec:ExecutionModel} and \ref{sec:MyFirstApplication}.

Given that these three operations are very common in class model definition (e.g. get/put operations in collections), we prepend the ``dc'' prefix to prevent an unexpected behavior due to potential overriden operations. Notice that \textit{get} is named as \textit{dcClone} to match OO terminology.

This section focuses on a set of static methods that can be called directly from \textit{DataClayObject} class.



\subsection{Class methods}
\label{sec:JavaClassMethodsObjectStore}

The following methods can be called from any downloaded stub as static methods. In this way, the user is allowed to access persistent objects by using their aliases (see Section~\ref{sec:JavaObjectStoreStubMethods} to see how objects are persisted with an alias assigned). Aliases prevent objects to be removed by the Garbage Collector, thus an operation to remove the alias of an object is also provided. More details on how the garbage collector works can be found in Section~\ref{sec:GarbageCollection}. 

Notice that the examples provided assume the initialization and finalization of user's session with methods described in previous section Section~\ref{sec:JavaGlobalAPI}.

% ------- dcCloneByAlias ---------

\begin{dBox}
\texttt{public static <T> \CALL{dcCloneByAlias} (String alias [, boolean recursive]) \newline throws DataClayException}
\LINE

{\it Description:}

\begin{itemize}
    \item Retrieves a copy of current object from dataClay. Fields referencing to other objects are kept as remote references to objects stored in dataClay, unless the recursive parameter is set to \textit{True}.
\end{itemize}

{\it Parameters:}
\begin{itemize}
    \item \texttt{\bfseries alias:} alias of the object to be retrieved.
    \item \texttt{\bfseries recursive:} When this is set to True, the default behavior is altered so not only current object but all of its references are also retrieved locally.
\end{itemize}

{\it Returns:}

\begin{itemize}
    \item A new object instance initialized with the field values of the object with the alias specified.
\end{itemize}

{\it Exceptions:}

\begin{itemize}
    \item If no object with specified alias exists, a DataClayException is raised.
\end{itemize}

\end{dBox}

\begin{tBox}
\textcolor{basecolor} {\bf Example: Using dcCloneByAlias method}
\begin{java}
Person newPerson = new Person("Alice", 32);
newPerson.dcPut("student1");
Person retrieved = Person.dcCloneByAlias("student1");
assertTrue(retrieved.getName().equals(newPerson.getName())
\end{java}
\end{tBox}


% ------- dcUpdateByAlias ---------

\begin{dBox}
\texttt{public static void \CALL{dcUpdateByAlias} (String alias, \newline DataClayObject fromObject) throws DataClayException}
\LINE

{\it Description:}

\begin{itemize}
    \item Updates the object identified by specified alias with contents of \textit{fromObject}.
\end{itemize}

{\it Parameters:}
\begin{itemize}
    \item \texttt{\bfseries alias:} alias of the object to be retrieved.
    \item \texttt{\bfseries fromObject:} the base object which contents will be used to update target object with alias specified.
\end{itemize}

{\it Exceptions:}

\begin{itemize}
    \item If no object with specified alias exists, a DataClayException is raised.
    \item If object identified with given alias has different fields than \textit{fromObject}, a DataClayException is raised.
\end{itemize}

\end{dBox}

\begin{tBox}
\textcolor{basecolor} {\bf Example: Using dcUpdateByAlias method}
\begin{java}
Person newP = new Person("Alice", 32);
newP.dcPut("student1");
Person newValues = new Person("Alice Smith", 35);
Person.dcUpdateByAlias("student1", newValues);
Person clonedP = Person.cloneByAlias("student1");
assertTrue(clonedP.getName().equals(newValues.getName()));
\end{java}
\end{tBox}


\subsection{Object methods}
\label{sec:JavaObjectStoreStubMethods}

This section expands Section~\ref{sec:JavaClassMethodsObjectStore} with methods that can be called directly from object stub instances. That is, stub classes are adapted to extend a common dataClay class called DataClayObject, which provides the following methods.


% ------- dcClone ---------

\begin{dBox}
\texttt{public <T> \CALL{dcClone} (([boolean recursive]) throws DataClayException}
\LINE

{\it Description:}

\begin{itemize}
    \item Retrieves a copy of current object from dataClay. Fields referencing to other objects are kept as remote references to objects stored in dataClay.
\end{itemize}

{\it Parameters:}
\begin{itemize}
    \item \texttt{\bfseries recursive:} When this is set to True, the default behavior is altered so not only current object but all of its references are also retrieved locally.
\end{itemize}

{\it Returns:}

\begin{itemize}
    \item A new object instance initialized with the field values of current object. Non-primitive fields or sub-objects are also copied by creating new objects.
\end{itemize}

{\it Exceptions:}

\begin{itemize}
    \item If current object is not persistent, a DataClayException is raised.
\end{itemize}

\end{dBox}

\begin{tBox}
\textcolor{basecolor} {\bf Example: Using dcClone method}
\begin{java}
Person p = new Person("Alice", 32);
p.dcPut("student1");
Person copy = p.dcClone()
assertTrue(copy.getAge() == p.getAge())
\end{java}
\end{tBox}


% ------- dcPut ---------

\begin{dBox}
\texttt{public void \CALL{dcPut} () throws DataClayException}\index{alias}

\texttt{public void \CALL{dcPut} (String alias [, BackendID backendID]) \newline throws DataClayException}\index{alias}

\texttt{public void \CALL{dcPut} (String alias [, boolean recursive]) \newline throws DataClayException}\index{alias}

\texttt{public void \CALL{dcPut} (String alias, BackendID backendID
\newline [, boolean recursive]) throws DataClayException}\index{alias}
\LINE

{\it Description:}

\begin{itemize}
    \item Stores an aliased object in the system and assigns an OID to it. Notice this method allows specifying a certain backend. In this regard, the \colorbox{basecolor!15}{\texttt{\bfseries DataClay.LOCAL}} field can be set as a constant for a specific backendID (as detailed in Section~\ref{sec:ClientConfigFiles}). To use this field from your application, you have to add the proper import: \colorbox{basecolor!15}{\texttt{import es.bsc.dataClay.api.DataClay}}.
\end{itemize}

{\it Parameters:}

\begin{itemize}
    \item \texttt{\bfseries alias:} a string that will identify the object in addition to its OID. Aliases are unique in the system.
    \item \texttt{\bfseries backendID:} identifies the backend where the object will be stored. If this parameter is missing, then a random backend is selected to store the object. When \texttt{DataClay.LOCAL} is used, the object is created in the backend specified as local in the client configuration file.
    \item \texttt{\bfseries recursive:} when this flag is True, all objects referenced by the current one will also be made persistent (in case they were not already persistent) in a recursive manner. When this parameter is not set, the default behavior is to perform a recursive makePersistent.
\end{itemize}

{\it Exceptions:}

\begin{itemize}
    \item If there is a stored object with the same alias, a DataClayException is raised.
    \item If a backend is specified and it is not valid, a DataClayException is raised. Use getBackends (\ref{call:JavaGetBackends}) to obtain valid backends.
\end{itemize}

\end{dBox}

\begin{tBox}
\textcolor{basecolor} {\bf Example: Using dcPut method}
\begin{java}
Person p = new Person("Alice", 32);
p.dcPut("student1", DataClay.LOCAL);
assertTrue(p.getLocation().equals(DataClay.LOCAL));
\end{java}
\end{tBox}


% ------- dcUpdate ---------

\begin{dBox}
\texttt{public void \CALL{dcUpdate} (DataClayObject fromObject) throws DataClayException}
\LINE

{\it Description:}

\begin{itemize}
    \item Updates current object with contents of \textit{fromObject}.
\end{itemize}

{\it Parameters:}
\begin{itemize}
    \item \texttt{\bfseries fromObject:} the base object which contents will be used to update target object with alias specified.
\end{itemize}

{\it Exceptions:}

\begin{itemize}
    \item If object to be updated is not persistent, a DataClayException is raised.
    \item If the object has different fields than \textit{fromObject}, a DataClayException is raised.
\end{itemize}

\end{dBox}

\begin{tBox}
\textcolor{basecolor} {\bf Example: Using dcUpdate method}
\begin{java}
Person p = new Person("Alice", 32);
p.dcPut("student1");
Person newValues = new Person("Alice Smith", 35);
p.dcUpdate(newValues);
assertTrue(p.getName().equals(newValues.getName()));
\end{java}
\end{tBox}



\section{Object oriented methods}
\label{sec:JavaObjectExtendedMethods}

Besides object-store operations, dataClay also offers a set of methods to enable applications work in a more Object-oriented fashion. 

In Object-oriented programming objects are connected by using navigable associations (object references). In dataClay, applications might have objects containing fields associated with other persistent objects through remote object references. Therefore, a set of extended methods are provided to expand object store methods presented in previous section \ref{sec:JavaObjectStore}.


\subsection{Class methods}
\label{sec:JavaClassMethods}

% ------- deleteAlias ---------

\begin{dBox}
\texttt{public static void \CALL{deleteAlias} (String alias) \newline throws DataClayException}\index{alias}
\LINE

{\it Description:}

\begin{itemize}
    \item Removes the alias linked to an object. If this object is not referenced starting from a root object and no active session is accessing it, the garbage collector will remove it from the system.
\end{itemize}


{\it Parameters:}

\begin{itemize}
    \item \texttt{\bfseries alias:} alias to be removed.
\end{itemize}

{\it Exceptions:}

\begin{itemize}
    \item If no object with specified alias exists, a DataClayException is raised.
\end{itemize}

\end{dBox}

\begin{tBox}
\textcolor{basecolor} {\bf Example: Using deleteAlias}
\begin{java}
Person newPerson = new Person("Alice", 32);
newPerson.makePersistent("student1");
...
Person.deleteAlias("student1");
\end{java}
\end{tBox}


% ------- getByAlias ---------

\begin{dBox}
\texttt{public static void <T> \CALL{getByAlias} (String alias) \newline throws DataClayException}\index{alias}
\LINE

{\it Description:}

\begin{itemize}
    \item Retrieves an object reference of current stub class corresponding to the persistent object with alias provided.
\end{itemize}

{\it Parameters:}

\begin{itemize}
    \item \texttt{\bfseries alias:} alias of the object.
\end{itemize}

{\it Exceptions:}

\begin{itemize}
    \item If no object with specified alias exists, a DataClayException is raised.
\end{itemize}

\end{dBox}

\begin{tBox}
\textcolor{basecolor} {\bf Example: Using getByAlias }
\begin{java}
Person newPerson = new Person("Alice", 32);
newPerson.makePersistent("student1");
Person refPerson = Person.getByAlias("student1");
assertTrue(newPerson.getName().equals(refPerson.getName())
\end{java}
\end{tBox}



\subsection{Object methods}

In Object-oriented programming, aliases are not required if we can refer to an object by following a navigable association from another object. Therefore, the following method is similar to \textit{dcPut} but offering the possibility to register an object without an alias.

% ------- makePersistent ---------

\begin{dBox}
\label{sec:JavaObjecMakePersistent}
\texttt{public void \CALL{makePersistent} () throws DataClayException}\index{alias}

\texttt{public void \CALL{makePersistent} (BackendID backendID) \newline throws DataClayException}\index{alias}

\texttt{public void \CALL{makePersistent} (String alias, [BackendID backendID]) \newline throws DataClayException}\index{alias}

\texttt{public void \CALL{makePersistent} (String alias, [boolean recursive]) \newline throws DataClayException}\index{alias}

\texttt{public void \CALL{makePersistent} (BackendID backendID, [boolean recursive]) \newline  throws DataClayException}\index{alias}

\texttt{public void \CALL{makePersistent} (String alias, BackendID backendID, 
\newline [boolean recursive]) throws DataClayException}\index{alias}
\LINE

{\it Description:}

\begin{itemize}
    \item Stores an object in dataClay and assigns an OID to it.
\end{itemize}

{\it Parameters:}

\begin{itemize}
    \item \texttt{\bfseries alias:} a string that will identify the object in addition to its OID. Aliases are unique in the system. If no alias is set, this object will not have an alias, will only be accessible though other object references.
    \item \texttt{\bfseries backendID:} identifies the backend where the object will be stored. If this parameter is missing, then a random backend is selected to store the object. When \texttt{DataClay.LOCAL} is used, the object is created in the backend specified as local in the client configuration file.
    \item \texttt{\bfseries recursive:} when this flag is True, all objects referenced by the current one will also be made persistent (in case they were not already persistent) in a recursive manner. When this parameter is not set, the default behavior is to perform a recursive makePersistent.
\end{itemize}

{\it Exceptions:}

\begin{itemize}
    \item If an alias is specified and there is a stored object with the same alias, a DataClayException is raised.
    \item If a backend is specified and it is not valid, a DataClayException is raised. Use getBackends (\ref{call:JavaGetBackends}) to obtain valid backends.
\end{itemize}

\end{dBox}

\begin{tBox}
\textcolor{basecolor} {\bf Example: Using makePersistent method}
\begin{java}
Person p = new Person("Alice", 32);
p.makePersistent("student1", DataClay.LOCAL);
assertTrue(p.getLocation().equals(DataClay.LOCAL));
\end{java}
\end{tBox}



\section{Advanced methods}
\label{sec:JavaObjectAdvanced}

In this section we present advanced methods that are also inherited from DataClayObject class. These methods are not intended to be used by standard programmers, but by runtime and library developers or expert programmers.


% ------- getAllLocations ---------

\begin{dBox}
\texttt{public Set<BackendID> \CALL{getAllLocations}() throws DataClayException}
\LINE

{\it Description:}

\begin{itemize}
    \item Retrieves all locations where the object is persisted/replicated.
\end{itemize}

{\it Returns:}

\begin{itemize}
    \item A set of backend IDs in which this object or its replicas are stored.
\end{itemize}

{\it Exceptions:}

\begin{itemize}
    \item If the object is not persistent, a DataClayException is raised.
\end{itemize}

\end{dBox}

\begin{tBox}
\textcolor{basecolor} {\bf Example: Using getAllLocations}
\begin{java}
Person p1 = Person.getByAlias("personalias");
Set<BackendID> locations = p1.getAllLocations();
if (!locations.contains(DataClay.LOCAL)) {
 p1.newReplica(DataClay.LOCAL);
}
\end{java}
\end{tBox}


% ------- getLocation ---------

\begin{dBox}
\texttt{public BackendID \CALL{getLocation}() throws DataClayException}
\LINE

{\it Description:}

\begin{itemize}
    \item Retrieves a location of the object. % The returned location is one randomly picked from all locations where the object or a replica are stored.
\end{itemize}
 
{\it Returns:}

\begin{itemize}
    \item Backend ID in which this object is stored. If the object is not persistent (i.e. it has never been persisted) this function will fail.
\end{itemize}

{\it Exceptions:}

\begin{itemize}
    \item If the object is not persistent, a DataClayException is raised.
\end{itemize}

\end{dBox}

\begin{tBox}
\textcolor{basecolor} {\bf Example: Using getLocation}
\begin{java}
Person p1 = Person.getByAlias("student1");
p1.makePersistent(DataClay.LOCAL);
assertTrue(p1.getLocation().equals(DataClay.LOCAL));
\end{java}
\end{tBox}


% ------- newReplica ---------

\begin{dBox}

\texttt{public BackendID \CALL{newReplica}() throws DataClayException}

\texttt{public BackendID \CALL{newReplica}(boolean recursive) throws DataClayException}

\texttt{public BackendID \CALL{newReplica}(BackendID backendID \newline [,boolean recursive]) throws DataClayException}
\LINE

{\it Description:}

\begin{itemize}
    \item Creates a replica of the current object.
    
    It is important to notice that dataClay does not take care of replica synchronization. Details on how such synchronization can be achieved are described in Section~\ref{sec:JavaReplication}.
    \newline
    \newline
    Notice that the replication of an object includes the replication of its subobjects (references) as the default behavior (i.e. recursive is True by default). Therefore, some objects (including current object) might be already present in the destination backend. These objects will be ignored from replication, since a backend cannot have two replicas of the same object. But it is ensured that, after a correct execution of this method, a full copy of the current object (and all its subobjects if recursive) is present in the returned backend (same as backendID if user specifies it).

\end{itemize}

{\it Parameters:}

\begin{itemize}
    \item \texttt{\bfseries backendID:} ID of the backend in which to create the replica. If null, a random backend is chosen. When \texttt{DataClay.LOCAL} is used, the object is replicated in the backend specified as local in the client configuration file.
    \item \texttt{\bfseries recursive:} when this flag is True, all objects referenced by the current one will also be replicated (except those that are already present in the destination backend). When this parameter is not set, the default behavior is to perform a recursive replica.
\end{itemize}

{\it Returns:}

\begin{itemize}
    \item The ID of the backend in which the replica was created. 
\end{itemize}

{\it Exceptions:}

\begin{itemize}
    \item If the object is not persistent, a DataClayException is raised.
    \item If a backend is specified and it is not valid, a DataClayException is raised. Use getBackends (\ref{call:JavaGetBackends}) to obtain valid backends.
\end{itemize}

\end{dBox}

\begin{tBox}
\textcolor{basecolor} {\bf Example: Using newReplica}
\begin{java}
Person p1 = Person.getByAlias("student1");
// replicating object and referenced objects
// from one of its locations to LOCAL
p1.newReplica(p1.getLocation(), DataClay.LOCAL);
\end{java}
\end{tBox}


% ------- runRemote --------

\begin{dBox}
\texttt{public Object \CALL{runRemote}(BackendID location, \newline String opID, Object[] params) throws DataClayException}
\LINE

{\it Description:}

\begin{itemize}
    \item Executes a specific method on a particular backend. Notice that currently this method is intended for synchronization purposes, as can be seen in
  section \ref{sec:JavaReplication}. Check that section for a proper example.
\end{itemize}

{\it Parameters:}

\begin{itemize}
    \item \texttt{\bfseries location:} Backend where the method must be executed. When \texttt{DataClay.LOCAL} is used, the execution request is sent to the backend specified as local in the client configuration file.
    \item \texttt{\bfseries opID:} ID of the method to be executed.
    \item \texttt{\bfseries params:} The regular parameters of the method.
\end{itemize}
 
{\it Returns:}

\begin{itemize}
    \item The expected result from the execution of the specified method.
\end{itemize}

{\it Exceptions:}

\begin{itemize}
    \item If this object is not persistent, a DataClayException is raised.
    \item If location specified is not valid, a DataClayException is raised. Use getBackends (\ref{call:JavaGetBackends}) to obtain valid backends.
\end{itemize}

\end{dBox}


% ------- moveObject ---------

%\begin{dBox}
%\texttt{public void \CALL{moveObject}(BackendID sBackendID, BackendID dBackendID \newline [,boolean recursive]) throws DataClayException}
%\LINE
%
%{\it Description:}
%
%\begin{itemize}
%  \item Moves the object (or replica) from a backend identified by sBackendID to another backend identified by dBackendID.
%\end{itemize}
%
%{\it Parameters:}
%
%\begin{itemize}
%  \item \texttt{\bfseries sBackendID:} ID of the source location in which the object is stored.
%  \item \texttt{\bfseries dBackendID:} ID of the destination location in which the object should be moved. 
%  \item \texttt{\bfseries recursive:} when this flag is True, all objects referenced by the current one will also be moved. When this parameter is not set, the default behavior is to perform a recursive move.
%\end{itemize}
%
%{\it Exceptions:}
%
%\begin{itemize}
%  \item If the object is not persistent, the object is not located int he source location, or any of the location IDs do not represent a valid location, a DataClayException is raised.
%\end{itemize}
% 
%\end{dBox}
%
%\begin{tBox}
%\textcolor{basecolor} {\bf Example: Using moveObject}
%\begin{lstlisting}
%Person p1 = Person.getByAlias("student1");
%// moving object, but not referenced objects, from one of its locations to LOCAL
%p1.moveObject(p1.getLocation(), DataClay.LOCAL, false);
%\end{lstlisting}
%\end{tBox}

\section {Error management}\index{error management}
\label{sec:JavaErrorHandling}

Besides \textit{DataClayException} raised from \textit{DataClayObject} methods or dataClay API methods as exposed along this chapter, exceptions raised from methods of your class models while running on a dataClay backend are also forwarded to end-user applications.

However, notice that current version of dataClay does not allow you to register your own exception classes (i.e. as part of your data model), so methods enclosed in your data model can only throw language built-in exceptions.

\section {Memory Management and Garbage Collection}\index{garbage collection}\index{memory management}

In section \ref{sec:GarbageCollection} we introduced the routines that aim to optimize memory and disk usage in the backends.

In Java, users cannot deallocate objects manually so in dataClay we do not provide a direct operation to do so. However, since we add an extra layer for persistence we have to ensure that the Java Garbage Collector (GC) does not remove loaded objects before they are synchronized with the underlying storage. To this end, a dataClay thread periodically checks if the memory usage reaches a certain threshold and, when this is the case, objects are firstly flushed to persistent storage in a way that Java GC can collect them.

On the other hand, a Global Garbage Collector keeps track of global reference counters in a per object basis. Considering the conditions that an object has to meet in order to be removed, as stated in section \ref{sec:GarbageCollection}, its associated reference counter not only counts which objects are pointing to it, but also how many aliases it has or the applications and running methods that are using it.

\section{Replica management}\index{replica management}
\label{sec:JavaReplication}

Given that each object or piece of data may potentially need a different consistency model, dataClay will not synchronize objects. On the other hand, it will offer mechanisms for the model developer to include it as part of the model in an easy way, and how to be able to import the consistency model form another class already defined.

The first way to guarantee the consistency level required by a replicated object is to add the needed code in all setters/getter of the class. Although this is a feasible option is quite impractical if we need to add this code to all classes we want to build. Fir this reason, dataClay also offers a mechanism to add arbitrary code (from a static class) to be executed before or after a given method. This mechanism, explained in detail in this section, will enable programmers to build their consistency model once (or use a predefined one) and use it in any of their classes without modifying the class itself.

In this section, we present how to add consistency code into existing classes.

Let us assume that we have our class Person:

\begin{tBox}
\begin{java}
public class Person {
  String name;
  int age;
  public Person(String name, int age) {
    this.name = name;
    this.age = age;
  }
}
\end{java}
\end{tBox}

Once this class is registered and with the proper permissions and stubs, an application that uses it might look like this:

\begin{tBox}
\begin{java}
public class App {
  public static void main(String[] args) {
    DataClay.init();
    Person p = new Person("Alice", 42);
    
    p.makePersistent("student1");
    p.newReplica();
    
    p.setAge(43);
    System.out.println(p.getAge());
  }
}
\end{java}
\end{tBox}

With no consistency policies, the printed message would show an unpredictable age for Alice, since methods \textit{setAge} and \textit{getAge} are executed in a random backend among the locations of the object.

In order to overcome this problem, dataClay provides a mechanism to define synchronization policies at user-level. In particular, class developers are allowed to define three different annotations to customize the behavior of attribute updates:

\begin{tBox}
\begin{java}
    @Replication.InMaster
    @Replication.BeforeUpdate(method="...", clazz="...")
    @Replication.AfterUpdate(method="...", clazz="...")
\end{java}
\end{tBox}

The \textit{InMaster} annotation forces the update operation to be handled from the master location. The default master location of an object is the backend where the object was originally stored.

On the other hand, \textit{BeforeUpdate} and \textit{AfterUpdate} define extra behavior to be executed before or after the update operation. The \textit{method} argument specifies an operation signature of a static class method. The \textit{clazz} argument refers to the class where such a static class method is implemented. In this way, the developer is allowed to define an action to be triggered before the update operation, and an action to be taken after the update operation.

Let us resume our previous example. Assuming that the \textit{name} attribute is never modified (e.g. private setter), we want, however, that every time the age is updated the change is propagated to all the replicas. Empowering Person class with the proper annotation, we can intervene updates of attribute \textit{age} to perform the update synchronization:

\begin{tBox}
\begin{java}
public class Person {
  String name;

  @Replication.InMaster
  @Replication.AfterUpdate(method="replicateToSlaves",
                                    clazz="model.SequentialConsistency")
  int age;
  public Person(String name, int age) {
    this.name = name;
    this.age = age;
  }
}
\end{java}
\end{tBox}

Following the example, and as part of the class model, the proposed \textit{SequentialConsistency} class can be implemented as follows:

\begin{tBox}
\begin{java}
package model;

import java.util.Set;

import api.BackendID;
import serialization.DataClayObject;

public class SequentialConsistency {
  public static void replicateToSlaves(DataClayObject o, String setter, Object[] args) {
    Set<BackendID> locations = o.getAllLocations();
    for (BackendID replicaLocation : locations) {
      if (!replicaLocation.equals(o.getMasterLocation())) {
         o.runRemote(replicaLocation, setter, args);
      }
    }
  }
}
\end{java}
\end{tBox}

In this example, the master replica leads a sequential consistency model by synchronizing the contents with secondary replicas. 

Some considerations merit the attention of model developers:

\begin{itemize}
    \item The master location of an object can be checked with the method \textit{getMasterLocation()}.
    \item The method name specified in the annotations is always implemented as a \textit{public static void} operation, which receives the context info about the original method that triggered the action. This context info consists of:
 \begin{itemize}
    \item A dataClay object reference. Object in which the original method is being executed. In our example, a reference to Person object.
    \item The method itself. An identifier that dataClay can manage. In our example, the \textit{setAge} method identifier.
    \item The arguments received by the method. In our example, the new age to be set.
 \end{itemize}
\end{itemize}

For convenience, the implementation of the \textit{SequentialConsistency} class in the example can be used by including the import:

\colorbox{basecolor!20}{\texttt{import es.bsc.dataclay.util.replication.Replication}}

\FEDERATION{
\section{Federation}\index{federation}
\label{sec:jFederation}

In some scenarios, such as edge-to-cloud environments, part of the data stored in a dataClay instance has to be shared with another dataClay instance running in a different device. An example can be found in the context of smart cities where, for instance, part of the data residing in a car is temporarily shared with the city the car is traversing. This partial, and possibly temporal, integration of data between independent dataClay instances is implemented by means of dataClay's federation mechanism.
More precisely, federation consists in replicating an object (either simple or complex, such as a collection of objects) in an independent dataClay instance so that the recipient dataClay can access the object without the need to contact the owner dataClay. This provides immediate access to the object, avoiding communications when the object is requested and overcoming the possible unavailability of the data source. 

An object can be federated with an unlimited number of other dataClay instances. Additionally, a dataClay instance that receives a federated object can federate it with other dataClay instances.

Federated objects can be synchronized in all dataClay instances sharing them, in such a way that only those parts of the data that change are transferred through the network in order to avoid unnecessary transfers. This is achieved analogously to the synchronization of replicas stored among different backends of a single dataClay, as explained below. 

To federate an object, both the source and the target dataClay must have the same data model registered. This is achieved by importing the model from the target dataClay, or from another dataClay instance holding the same model as the target dataClay. This process is done through the methods \textit{RegisterDataClay} and \textit{ImportModelsFromExternalDataClay} (as well as the usual \textit{GetStubs}) before the execution of the application (see Section \ref{sec:dClayTool}).

In this section we present how to manage federation of objects that instantiate Java classes. 

Assume we have our class Person:

\begin{tBox}
\begin{java}
public class Person {
  String name;
  int age;
  public Person(String name, int age) {
    this.name = name;
    this.age = age;
  }
}
\end{java}
\end{tBox}

An application that federates an object of this class with another dataClay might look like this:

\begin{tBox}
\begin{java}
import es.bsc.dataclay.api.DataClay;

public class App {
  public static void main(String[] args) {
    DataClay.init();
    otherDC = DataClay.registerDataClay(host, port)
    
    Person p = new Person("Alice", 42);
    p.makePersistent("person1");
    
    p.federate(otherDC); 
  }
}
\end{java}
\end{tBox}

The first step is to make both dataClay instances aware of each other by means of the \textit{registerDataClay} method, explained in section \ref{sec:JavaFederationAPI}. The dataClay instance id returned by this call is used as a parameter for the \textit{federate} call on the object to indicate the dataClay instance that will receive the federated object. As explained above, note that both dataClay instances must have the same data model registered. 
At this point, an application accessing the dataClay instance \textit{otherDC} can execute the following code:

\begin{tBox}
\begin{java}
public class App {
  public static void main(String[] args) {
    DataClay.init();
    
    Person p = Person.getByAlias("person1");
    
    System.out.println(p.name);
  }
}
\end{java}
\end{tBox}

The secondary dataClay has actually performed a replica of Person object aliased \textit{person1}. From now on, this 
replica can be used in the execution environment of any of the backends of the secondary dataClay, as any other object created in \textit{otherDC}.

A user-defined behaviour can optionally be attached to the class of the object to be federated, which will be executed upon reception of the object in the target dataClay instance. To do this, a method \textit{whenFederated} must be implemented in the corresponding class, for instance:

\begin{tBox}
\begin{java}
public class Person {
  String name;
  int age;
 
 public Person(String name, int age) {
    this.name = name;
    this.age = age;

 public whenFederated() {
    PersonList pl = PersonList.getByAlias("persons"); 
    pl.add(this);
  }
}
\end{java}
\end{tBox}

In this way, the application accessing the target dataClay instance can use the collection \textit{pl} to get all the available objects of class \textit{Person} at any time. Notice that \textit{pl} is not a federated object, but a collection residing in the target dataClay instance that includes objects federated from the source dataClay (as well as possibly other objects created in the target dataClay instance).

\begin{tBox}
\begin{java}
public class App {
  public static void main(String[] args) {
    ...
    
    PersonList pl = new PersonList();
    pl.makePersistent("persons");
    ...
    length = pl.size();
    ...
  }
}
\end{java}
\end{tBox}
 
Federated objects can be synchronized using the same mechanisms provided to synchronize replicas within a dataClay instance, as explained in \ref{sec:JavaReplication}. To implement customized synchronization mechanisms on federated objects, the methods to be used are \textit{getFederationTargets}, which returns the identifiers of the dataClay instances where the object is federated, and \textit{getFederationSource}, which returns the source dataClay instance of a federated object in the current dataClay. Also, the method \textit{setInBackend} is provided to execute a setter method on the replica of the object that is stored in the specified dataClay instance. The description of these methods can be found in section \ref{sec:JavaFederationObject}.

For convenience, to synchronize federated objects following a sequential consistency policy, the method \textit{synchronizeFederated} in the same \textit{SequentialConsistency} class can be used.

Both the source and the target dataClay instance can stop sharing an object by calling the \textit{unfederate} method on the federated object. Then, the replica in the target dataClay will be eventually removed by the garbage collector unless it has an alias or it is referenced by another object. In any case, it will cease to be synchronized with the original object. 

Analogously to federation, the method \textit{whenUnfederated} can be implemented in the corresponding class to execute a customized behaviour in the target dataClay instance when an object is unfederated (for instance, removing the object from the \textit{PersonList} in the example above, so that the object can be garbage-collected.

\FEDERATION{
\section{Federation mechanism}
\label{sec:JavaObjectFederation}

More details about federation are presented in Section~\ref{sec:jFederation}.

% ------- federate ---------
\begin{dBox}
\texttt{public void \CALL{federate}(DataClayInstanceID dcID [,boolean recursive])}
\LINE

{\it Description:}

\begin{itemize}
  \item Federates current object with another dataClay instance. 
\end{itemize}

{\it Parameters:}

\begin{itemize}
  \item \texttt{\bfseries dcID:} ID of the external dataClay. It must be previously registered.
  \item \texttt{\bfseries recursive:} when this flag is TRUE, all objects (recursively) referenced by the current one will also be federated (except those that are already present in the destination dataClay). This parameter is optional, default value is TRUE.
\end{itemize}

\end{dBox}

\begin{tBox}
\textcolor{basecolor} {\bf Example: Using federate}
\begin{java}
DataClayID otherDC = DataClay.getDataClayID(host, port);
Person p1 = Person.getByAlias("person1");
// federating object and subobjects to otherDC (previously registered)
p1.federate(otherDC);
\end{java}
\end{tBox}

% ------- getFederationSource ------

\begin{dBox}
\texttt{public DataClayInstanceID \CALL{getFederationSource}()}
\LINE

{\it Description:}

\begin{itemize}
 \item Retrieves the ID of the dataClay instance where the object is federated from. 
\end{itemize}

{\it Returns:}

\begin{itemize}
 \item A DataClayInstanceID which is the source of this federated object.  
 It is null if the object is not federated.
\end{itemize}

\end{dBox}

% ------- getFederationTargets ------

\begin{dBox}
\texttt{public Set<DataClayInstanceID> \CALL{getFederationTargets}()}
\LINE

{\it Description:}

\begin{itemize}
 \item Retrieves the IDs of all the dataClay instances where the object is federated to. 
\end{itemize}

{\it Returns:}

\begin{itemize}
 \item A set of DataClayInstanceID objects in which this object is federated. 
 It is empty if the object is not federated.
\end{itemize}

\end{dBox}

\begin{tBox}
\textcolor{basecolor} {\bf Example: Using getFederationTargets}
\begin{java}
Person p1 = Person.getByAlias("personalias");
// using getFederationTargets to check if p1 is federated
Set<DataClayInstanceID> federation = p1.getFederationTargets();
return (!federation.size() == 0)
\end{java}
\end{tBox}

% ------- synchronizeFederated --------

\begin{dBox}
\texttt{public void \CALL{synchronizeFederated}(DataClayInstanceID dcID, \newline ImplementationID implID, Object[] params)} 
\LINE

{\it Description:}

\begin{itemize}
  \item Executes an implementation on a particular dataClay where the object is federated, for synchronization purposes.
\end{itemize}

{\it Parameters:}

\begin{itemize}
  \item \texttt{\bfseries dcID:} dataClay instance where the method must be executed.
  \item \texttt{\bfseries pimlID:} ID of the implementation to be executed.
  \item \texttt{\bfseries params:} The parameters of the method.
\end{itemize}
 
\end{dBox}

% ------- unfederate ---------
\begin{dBox}

\texttt{public void \CALL{unfederate}([DataClayInstanceID dcID] [,boolean recursive])}
\LINE

{\it Description:}

\begin{itemize}
  \item Unfederates current object (and referenced objects) with the indicated dataClay instance. If no dataClayID is specified, the object is unfederated from all the instances where it lives.
\end{itemize}

{\it Parameters:}

\begin{itemize}
  \item \texttt{\bfseries dcID:} ID of the external dataClay. It must be previously registered.
  \item \texttt{\bfseries recursive:} when this flag is TRUE, all objects (recursively) referenced by the current one will also be unfederated. This parameter is optional, default value is TRUE.
\end{itemize}

\end{dBox}


}
}

\section{Further considerations}
\label{sec:JavaConsiderations}

This section exposes some particularities that are coupled to current dataClay requirements or limitations.

% Current version of dataClay has some unsupported features for Java class models. Most significant ones are listed below.
% 
% \begin{enumerate}
%  \item Non-final static attributes or class attributes. Only final static attributes are supported.
%  \item User-defined interfaces. Only language predefined interfaces can be implemented.
%  \item Non-distributed \texttt{synchronized}. You can code a \texttt{synchronized} method or statement, but they will be only effective in a per backend basis. That is, \texttt{synchronize} on a method or object will not behave as a distributed lock.
%  \item Assertions. Due to its exceptional behavior, currently we are not providing \texttt{assert} support since it will require special flags to execute JVM on dataClay backends.
%  \item Third-party libraries. You can assume that classes from JDK 1.8 (i.e. \textit{java.util.*}) are available for your data models, but current version of dataClay does not support the registration of external jar files.
%  \item Handover of lambda functions. Lambdas can only be used in the context of a single execution environment. That is, lambdas cannot be passed in remote execution requests.
%  \item User-defined exceptions. You can throw Java exceptions from methods of your registered class models, but you cannot define your own Exceptions as part of your class model.
%  \item Handover of non-serializable Java classes. They can be used within a method (e.g. Iterators), but cannot be passed as argument or returned as result.
% \end{enumerate}


\subsection{Importing registered classes}
\label{sec:JavaImports}

In order to use certain classes from registered data models you will have to specify the imports for the corresponding stubs. To this end, you have to ensure that Java \textit{classpath} includes the path of your stubs directory when running your applications.

\subsection{Non-registered classes}
Non-registered mutable types (such as Java built-in collections) are opaque to dataClay. Thus, when a registered class has one of such objects (as a field) and this mutable object is modified from outside its containing class, the changes in the mutable object may not be reflected.

For example, given a Class A with a field b of type B, and B has a field list of type \texttt{ArrayList}. After executing the instruction \texttt{this.b.list.add(x)} from a method in A, the list may not contain the new element x. To solve this, the class model should define a method in class B containing the instruction \texttt{b.list.add(x)} and call it from class A.

\subsection{Third party libraries}

Sometimes using third-party libraries from registered data models is not trivial, thus if you experience such problems, please contact us by email: \texttt{\href{mailto:support-dataclay@bsc.es}{support-dataclay@bsc.es}}

\FILTERING{
\section{Collection filtering}\index{filtering}
\label{sec:JavaFiltering}

\TODO{Check it makes sense}
\TODO{Python filtering}
\TODO{Check Grammar is correct}
\TODO{Maybe we could decide a unique method for filtering, currently we have two or more for non-AST queries}

A user class implementing the \textit{Iterable} interface can benefit from filtering methods provided by DataClay objects (i.e. from the corresponding stubs).

The method to filter an iterable object, i.e. a collection, only requires a query in String format compliant with the following grammar:

\begin{lstlisting}
 Filter  ::= ( AndExpr )or( Filter )*
 
 AndExpr ::= ( Comp )and( AndExpr )*
 
 Comp    ::= Attribute Op Value
         | Value Op Attribute
         | Filter
 
 Op      ::= '<' | '<=' | '=' | '>=' | '>' | '!=' | ':=' | '^='
 
 Attribute ::= ? attribute name possibly nested ?
 
 Value ::=IntValue | DateValue | StringValue | BoolValue

 IntValue ::= /[0-9]+/ | -/[0-9]+/
 DateValue ::= ? in Java standard String format ?
 StringValue ::= "..." | '...'
 BoolValue ::= 'true' | 'false'
\end{lstlisting}

The following example illustrates a basic class model representing an iterable collection of Person objects (following HelloPeople example in section \ref{sec:HelloPeople}) to filter them afterwards.

\begin{tBox}
\texttt{\bfseries\textcolor{basecolor}{HelloPeople.java}}
\begin{lstlisting}[language=Java]
package model;
import model.Person;
import java.util.Iterator;
import java.util.List;
import java.util.ArrayList;

public class PersonList implements Iterable<Person> {
    public List<Person> people;

    public People() {
      people = new ArrayList<>();
    }
    
    @Override
    public Iterator<Person> iterator() {
      return people.iterator();
    }
}

\end{lstlisting}
\end{tBox}

This basic class could be filled with Person objects and once registered and obtained the corresponding stub, you will be enabled to execute the following method.

\begin{dBox}
\texttt{public List<Object> \CALL{filterStream}(final String conditions)}
\LINE

{\it Description:}

\begin{itemize}
  \item If 'this' object is iterable (i.e. its class implements Iterable interface), the query is applied by iterating through the object and checking the provided conditions. Objects found matching these conditions are returned in a List.
\end{itemize}

{\it Parameters:}

\begin{itemize}
  \item \texttt{\bfseries conditions:} The query conditions matching the previously presented grammar.
\end{itemize}
 
{\it Returns:}

\begin{itemize}
  \item The list of objects that match the conditions specified within the query.
\end{itemize}

{\it Exceptions:}

\begin{itemize}
  \item If this object is not Iterable.
\end{itemize}

\end{dBox}
}
