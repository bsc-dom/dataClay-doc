\chapter{Java API security}
\label{sec:JavaAPI security}

\subsection{dataClay federation with secure communications}

In this section we explain how to secure dataClay communications between different dataClay instances.

In a federated environment, different dataClays are communicating to each other via LogicModule service. 

Current implementation of dataClay does not allow you to specify certificates in server-side but only client certificates. 
Thus, we use a Traefik reverse-proxy \href {https://docs.traefik.io/} {https://docs.traefik.io/} 
to check client certificates and also to not publish dataClay ports. 

An example of the \textit{docker-compose.yml} file with reverse-proxy:


\begin{tBox}
 \begin{lstlisting}[language=docker-compose-2, frame=none]
version: '3.4'
services:

  proxy:
    image: traefik:v1.7.17
    restart: unless-stopped
    command: --docker --docker.exposedByDefault=false
    volumes:
      - /var/run/docker.sock:/var/run/docker.sock:ro
      - /home/docker/common/dataclay/traefik.toml:/traefik.toml
      - /home/docker/certs:/ssl:ro
    ports:
      - "80:80"
      - "443:443"
      
  logicmodule:
    image: "bscdataclay/logicmodule:2.0"
    environment:
      - LOGICMODULE_PORT_TCP=11034
      - LOGICMODULE_HOST=logicmodule
      - DATACLAY_ADMIN_USER=admin
      - DATACLAY_ADMIN_PASSWORD=admin
    volumes:
      - ./prop/global.properties:/usr/src/dataclay/javaclay/cfgfiles/global.properties:ro
      - ./prop/log4j2.xml:/usr/src/dataclay/javaclay/log4j2.xml:ro
    healthcheck:
       interval: 5s
       retries: 10
       test: ["CMD-SHELL", "/usr/src/dataclay/javaclay/health_check.sh"]
    labels:
      - "traefik.enable=true"
      - "traefik.backend=logicmodule"
      - "traefik.frontend.rule=Headers: service-alias,logicmodule"
      - "traefik.port=11034"
      - "traefik.protocol=h2c"
          
  dsjava:
    image: "bscdataclay/dsjava:2.0"
    depends_on:
      - logicmodule
    environment:
      - DATASERVICE_NAME=DS1
      - DATASERVICE_JAVA_PORT_TCP=2127
      - LOGICMODULE_PORT_TCP=11034
      - LOGICMODULE_HOST=logicmodule
    volumes:
      - ./prop/global.properties:/usr/src/dataclay/javaclay/cfgfiles/global.properties:ro
      - ./prop/log4j2.xml:/usr/src/dataclay/javaclay/log4j2.xml:ro
    healthcheck:
       interval: 5s
       retries: 10
       test: ["CMD-SHELL", "/usr/src/dataclay/javaclay/health_check.sh"]
       
  dspython:
    image: "bscdataclay/dspython:2.0"
    depends_on:
      - logicmodule
      - dsjava
    environment:
      - DATASERVICE_NAME=DS1
      - LOGICMODULE_PORT_TCP=11034
      - LOGICMODULE_HOST=logicmodule
      - DATASERVICE_PYTHON_PORT_TCP=6867
    volumes:
      - ./prop/global.properties:/usr/src/dataclay/pyclay/cfgfiles/global.properties:ro
    healthcheck:
       interval: 5s
       retries: 10
       test: ["CMD-SHELL", "/usr/src/dataclay/pyclay/health_check.sh"]
 \end{lstlisting}
\end{tBox}

With following \textit{traefik.toml} example:

\begin{tBox}
 \begin{lstlisting}[language=docker-compose-2, frame=none]
debug = false
defaultEntryPoints = ["http", "https"]
[entryPoints]
    [entryPoints.http]
    address = ":80"
        [entryPoints.http.redirect]
        entryPoint = "https"

   [entryPoints.https]
     address = ":443"
   [entryPoints.https.tls]
     [entryPoints.https.tls.clientCA]
     files = ["/ssl/dataclay-ca.crt"]
     optional = false

	 [entryPoints.https.tls.defaultCertificate]
     certFile = "/ssl/dataclay-agent.crt"
     keyFile  = "/ssl/dataclay-agent.pem"
     
     # For secure connection on frontend.local
     [[entryPoints.https.tls.certificates]]
     certFile = "/ssl/dataclay-agent.crt"
     keyFile  = "/ssl/dataclay-agent.pem"
 \end{lstlisting}
\end{tBox}

Note that ports are not published in \textit{docker-compose.yml} and we configure Traefik by adding labels to logicmodule service.

Finally, we need to configure our application to use the certificates. For that we have the following \textit{global.properties} options:

\begin{table}[H]
\footnotesize
\begin{tBox}
\centering
\begin{tabular}{p{57mm} | p{27mm} |  >{\raggedright\arraybackslash}p{50mm}}
\textbf{property} & \textbf{default value} & \textbf{description} \\
\hline
\verb LM_SERVICE_ALIAS_HEADERMSG & logicmodule & Add to the message the header service-alias (used to filter in traefik) . \\
\hline
\verb SSL_TARGET_AUTHORITY & proxy  & Override target authority (usually traefik service name). \\
\hline
\verb SSL_CLIENT_TRUSTED_CERTIFICATES & None & Path to CA certificate. \\
\hline
\verb SSL_CLIENT_CERTIFICATE & None & Path to Client certificate. \\
\hline
\verb SSL_CLIENT_KEY & None & Path to Client key. \\
\end{tabular}
\label{table:Security}
\end{tBox}
\end{table}

An example of the \textit{global.properties} file with TLS could be

\begin{tBox}
 \begin{bash}
LM_SERVICE_ALIAS_HEADERMSG=logicmodule
SSL_TARGET_AUTHORITY=proxy 
SSL_CLIENT_TRUSTED_CERTIFICATES=/usr/src/demo/app/certs/dataclay-ca.crt
SSL_CLIENT_CERTIFICATE=/usr/src/demo/app/certs/dataclay-agent.crt
SSL_CLIENT_KEY=/usr/src/demo/app/certs/dataclay-agent.pem
 \end{bash}
\end{tBox}