\chapterimage{Client.jpg} % Chapter heading image

\chapter{Hands-on Examples}\index{HelloPeople}
\label{sec:FullDemo}

This section explains how to run the \texttt{HelloPeople} applications (introduced in section \ref{sec:HelloPeople}) with a minimal docker-based installation in your own machine (as explained in section \ref{sec:SystemInstall}).

The full process is based on Linux-based Bash scripts and both Java and Python versions of the application will be tested.

\section{Requirements}

To run the tests, your system needs to fulfill the following requirements:

\begin{enumerate}
 \item Java 1.8: to run the Java version of HelloPeople and the dClayTool.
 \item Python 3: to run Python version of HelloPeople.
 \item Python pip: to install Python client library.
 \item Docker support: to launch a Docker-based installation of dataClay including:
 \begin{itemize}
    \item docker tools
    \item docker-compose
 \end{itemize}
\end{enumerate}

\section{Download}
\label{sec:DownloadZIP}

Start downloading the zip file from our website (Downloads tab):\newline

\footnotesize\url{https://www.bsc.es/research-and-development/software-and-apps/software-list/dataclay}\newline

Extract it into, for example, a directory named \texttt{dataClay} and follow next steps.

\section{Zip contents}

After extracting the contents of the zip file you will find the following directories in \texttt{dataClay} folder:
\begin{tBox}
\begin{bash}
  examples/  apps including HelloPeople and its class model.
  lib/       the Java client lib: dataClay.jar and dependencies.
  tools/     dClayTool used to register class models.
  
  containers/docker/         docker-compose examples to orchestrate dataClay services with Docker
  containers/singularity/    singularity setup example
\end{bash}
\end{tBox}

Notice that Python client library is not included since it will be installed directly from \textit{pip} repositories.

\section{Check and install}

At this point you can run the main script that checks and installs dataClay:

\begin{tBox}
 \begin{bash}
  > cd dataClay
  > bash install.sh
 \end{bash}
\end{tBox}

This script checks that your system has proper versions of Java and Python, as well as docker support. It also retrieves our Docker images from Docker Hub if necessary and installs the Python client library from the Python Package Index (\textit{PyPI}) using \texttt{pip} command.

Analogously, you can also uninstall dataClay:
\begin{tBox}
 \begin{bash}
  > cd dataClay
  > bash uninstall.sh
 \end{bash}
\end{tBox}

\section{Run HelloPeople}

Now everything is ready to run your first dataClay application either in Java or Python. If you navigate through \texttt{examples} directory you will find that \texttt{HelloPeople} folder contains both the Java and Python implementations.

To run the Java version with Object Store methods:

\begin{tBox}
 \begin{bash}
  > cd dataClay/examples/HelloPeople/java
  > bash demo_os.sh
 \end{bash}
\end{tBox}

To run the Java version with Object Oriented methods:

\begin{tBox}
 \begin{bash}
  > cd dataClay/examples/HelloPeople/java
  > bash demo_oo.sh
 \end{bash}
\end{tBox}

To run the Python version with Object Store methods:

\begin{tBox}
 \begin{bash}
  > cd dataClay/examples/HelloPeople/python
  > bash demo_os.sh
 \end{bash}
\end{tBox}

To run the Python version with Object Oriented methods:

\begin{tBox}
 \begin{bash}
  > cd dataClay/examples/HelloPeople/python
  > bash demo_oo.sh
 \end{bash}
\end{tBox}

\section{Dissecting demo scripts}

If you want to take a look at the \texttt{demo\_*.sh} scripts used to execute HelloPeople both in Java and Python, you will basically find these following sections:

\begin{enumerate}
 \item Docker initialization. dataClay containers are initialized with \texttt{docker-compose} command.
 \item Management operations. Basic user information is registered in the system as well as the application class model. The \texttt{dClayTool} is used to this end.
 \item Stubs retrieval. The \texttt{dClayTool} is now used to retrieve the stub classes corresponding to previously registered model and stores them in \texttt{stubs} directory.
 \item App compilation. HelloPeople is now compiled using the downloaded stubs and its classes are stored in \texttt{bin} directory. This is step is not necessary in Python.
 \item App execution. HelloPeople is executed.
 \item Shut down. dataClay containers are shut down with \texttt{docker-compose} command.
\end{enumerate}

Notice that application examples also contain a \texttt{cfgfiles} directory including the \texttt{session.properties} and the \texttt{client.properties} files. As exposed in section \ref{sec:ClientConfigFiles} these files are required to establish the connection and initialize a session with dataClay.

\section{Other examples}
\label{sec:FullTestOtherExamples}

Besides HelloPeople, in the examples directory, you will find another two well-known applications with their corresponding class models: K-means and Wordcount. K-means is a data clustering algorithm commonly used for machine learning (\url{http://en.wikipedia.org/wiki/K-means\_clustering}) and Wordcount is a simple application that counts all the occurrences for each unique word within a set of texts.

These applications not only benefit from dataClay persistence and its runtime environment, but also from its integration with the COMP Superscalar programming model \cite{tejedor2017pycompss} \cite{BADIA201532} for task parallelization. You can download COMP Superscalar from this link: 

{\begin{center} \scriptsize \url{https://www.bsc.es/research-and-development/software-and-apps/software-list/comp-superscalar} \end{center}}

In the case of Python applications, parallelism can be exploited as introduced in section \ref{sec:DataService} by running multiple \textit{Execution Environments} on a single node. A specific docker-compose file to orchestrate this kind of scenario is also provided.

If you want to learn more about these examples or need help with your own applications, please contact us: \texttt{\href{mailto:support-dataclay@bsc.es}{support-dataclay@bsc.es}}

\section{Singularity}\index{singularity}

dataClay can also be deployed using Singularity by converting our Docker images into Singularity ones. We provide an example to perform this setup and orchestrate Singularity containers in the directory \texttt{containers/singularity} of the downloadable zip (introduced in Section~\ref{sec:DownloadZIP}). Current supported Singularity version is $\geq$2.4.2.

\section{POM based projects}
\label{sec:POMbasedProjects}

If you are deploying pom-based applications you can add the following dependency into your pom file to install the Java client library:

\begin{tBox}
\footnotesize
 \begin{bash}
  <dependency>
    <groupId>dataclay</groupId>
    <artifactId>dataclay</artifactId>
    <version>1.0</version>
  </dependency>
 \end{bash}
\end{tBox}

Current version of dataClay is not hosted through Maven central repository, so you also have to add the following repository to resolve previous dependency from our GitHub repository:

\begin{tBox}
\footnotesize
 \begin{bash}
  <repository>
    <id>DataClay repository</id>
    <url>https://raw.github.com/bsc-ssrg/dataclay-maven/repository</url>
  </repository>
 \end{bash}
\end{tBox}
