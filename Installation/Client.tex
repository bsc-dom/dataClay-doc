\chapterimage{Client.jpg} % Chapter heading image

\chapter{Configuration}

\section{Client libraries}
\label{sec:ClientLibraries}

In order to connect your applications with dataClay services you need a client library for your preferred programming language.

If you are developing a Java application you can obtain the library following instructions in \ref{sec:FullDemo} (downloading our zip file as exposed in Section~\ref{sec:DownloadZIP} or following instructions for POM-based projects in Section~\ref{sec:POMbasedProjects}).

In case you are developing a Python application, you can easily install the Python module with \textit{pip} command:

\begin{tBox}
\begin{bash}
> pip install dataClay
\end{bash}
\end{tBox}

\section{Configuration files}
\label{sec:ClientConfigFiles}

The basic client configuration for an application is the minimum information required to initialize a session with dataClay. To this end two different files are required: the \textit{session.properties}\index{session.properties} file and the \textit{client.properties}\index{client.properties} file.

\subsection{Session properties}
This file contains the basic info to initialize a session with dataClay. It is automatically loaded during the initialization process (\texttt{DataClay.init()}\index{DataClay.init()} in Java or \texttt{api.init()}\index{api.init()} in Python) and its default path is \texttt{./cfgfiles/session.properties}. This path can be overridden by setting a different path through the environment variable \texttt{DATACLAYSESSIONCONFIG}. 

Here is an example:

\begin{tBox}
 \begin{bash}
  Account=MyAccount
  Password=MyPassword
  StubsClasspath=/home/me/myapp/stubs
  DataSetForStore=MyDataset
  DataSets=MyDataset,OtherDataSet
  LocalBackend=DS1
 % DataClayClientConfig=/home/me/myapp/client.properties
 \end{bash}
\end{tBox}

\texttt{Account} and \texttt{Password} properties are used to specify user's credentials. 

\texttt{StubsClasspath} defines a path where the stub classes can be located. That is, the path where \textit{dClayTool} (exposed in section \ref{sec:dClayTool}) saved our stub classes after calling \texttt{GetStubs} operation.

\texttt{DataSetForStore} specifies which dataset the application will use in case a \textit{makePersistent} request is produced to store a new object in the system, and \texttt{DataSets} provide information about the datasets the application will access (normally it includes the \texttt{DataSetForStore}). 

\texttt{LocalBackend} defines the default backend that the application will access when using either \texttt{DataClay.LOCAL} in Java or \texttt{api.LOCAL} in Python (examples of this can be found in API sections \ref{sec:JavaAPI} and \ref{sec:PythonAPI}). 

%Finally, the \texttt{DataClayClientConfig} contains a path pointing to \textit{client.properties} file, which is the second file required as exposed in next section.

\subsection{Client properties}
This file contains the minimum service info to connect applications with dataClay. It is also loaded automatically during the initialization process and its default path is \texttt{./cfgfiles/client.properties}, which can be overriden by setting the environment variable \texttt{DATACLAYCLIENTCONFIG}.

Here is an example:

\begin{tBox}
 \begin{bash}
 HOST=localhost
 TCPPORT=11034
 \end{bash}
\end{tBox}

As you can see, it only requires two properties to be defined: \texttt{HOST} and \texttt{TCPPORT}; comprising the full address to be resolved in order to initialize a session with dataClay from your application.

\section{Tracing}

dataClay provides a built-in tracing system to generate post-execution tracefiles of the applications’ execution. 
This is achieved using Extrae (\href {https://tools.bsc.es/extrae} {https://tools.bsc.es/extrae}).

For each service, Extrae keeps track of the events in an intermediate format file (with .mpit extension). At the end of the execution, all intermediate
files are gathered and merged by Extrae’s in order to create the final
tracefile, a Paraver format file (.prv) (\href {https://tools.bsc.es/paraver} {https://tools.bsc.es/paraver})

In order to enable Extrae tracing in dataClay, the application must activate it. We must write \texttt{Tracing=True} in the application's session.properties file:

\begin{tBox}
 \begin{bash}
  Account=MyAccount
  Password=MyPassword
  StubsClasspath=/home/me/myapp/stubs
  DataSetForStore=MyDataset
  DataSets=MyDataset,OtherDataSet
  LocalBackend=DS1
  Tracing=True
 \end{bash}
\end{tBox}

Also, we need to modify dataClay's \texttt{docker-compose.yml} to add \texttt{--tracing} command: 

\begin{tBox}
 \begin{lstlisting}[language=docker-compose-2, frame=none]
version: '3.4'
services:
  logicmodule:
    image: "bscdataclay/logicmodule:2.0"
    command: --tracing
    ports:
      - "11034:11034"
    environment:
      - LOGICMODULE_PORT_TCP=11034
      - LOGICMODULE_HOST=logicmodule
      - DATACLAY_ADMIN_USER=admin
      - DATACLAY_ADMIN_PASSWORD=admin
    volumes:
      - ./prop/global.properties:/usr/src/dataclay/javaclay/cfgfiles/global.properties:ro
      - ./prop/log4j2.xml:/usr/src/dataclay/javaclay/log4j2.xml:ro
    healthcheck:
       interval: 5s
       retries: 10
       test: ["CMD-SHELL", "/usr/src/dataclay/javaclay/health_check.sh"]
         
  dsjava:
    image: "bscdataclay/dsjava:2.0"
    command: --tracing
    ports:
      - "2127:2127"
    depends_on:
      - logicmodule
    environment:
      - DATASERVICE_NAME=DS1
      - DATASERVICE_JAVA_PORT_TCP=2127
      - LOGICMODULE_PORT_TCP=11034
      - LOGICMODULE_HOST=logicmodule
    volumes:
      - ./prop/global.properties:/usr/src/dataclay/javaclay/cfgfiles/global.properties:ro
      - ./prop/log4j2.xml:/usr/src/dataclay/javaclay/log4j2.xml:ro
    healthcheck:
       interval: 5s
       retries: 10
       test: ["CMD-SHELL", "/usr/src/dataclay/javaclay/health_check.sh"]
       
  dspython:
    image: "bscdataclay/dspython:2.0"
    command: --tracing
    depends_on:
      - logicmodule
      - dsjava
    environment:
      - DATASERVICE_NAME=DS1
      - LOGICMODULE_PORT_TCP=11034
      - LOGICMODULE_HOST=logicmodule
    volumes:
      - ./prop/global.properties:/usr/src/dataclay/pyclay/cfgfiles/global.properties:ro
    healthcheck:
       interval: 5s
       retries: 10
       test: ["CMD-SHELL", "/usr/src/dataclay/pyclay/health_check.sh"]
 \end{lstlisting}
\end{tBox}

Now we can start dataClay and run our application with tracing. Once finished, traces will be generated and stored in \texttt{`pwd`/traces} directory. 
Those traces are ready to be used by Paraver tool (\href {https://tools.bsc.es/paraver} {https://tools.bsc.es/paraver})

dataClay extrae traces can be used together with COMPSs (\href {https://compss.bsc.es} {https://compss.bsc.es}) 

Each node/service has an Extrae Task ID defined. Task ID is used to define different threads and lines in Paraver visualization. It means that in COMPSs you will have defined task IDs for master and workers (task ID = 0 for master, task ID = 1 for first worker, task ID = 2 for second worker, \ldots)

dataClay needs to use the first available task ID which is \texttt{task ID = COMPSs workers + 1}

Modify session.properties file and add the option \texttt{ExtraeStartingTaskID=taskID} to define the first available task ID for dataClay

\begin{tBox}
 \begin{bash}
  Account=MyAccount
  Password=MyPassword
  StubsClasspath=/home/me/myapp/stubs
  DataSetForStore=MyDataset
  DataSets=MyDataset,OtherDataSet
  LocalBackend=DS1
  Tracing=True
  ExtraeStartingTaskID=2
 \end{bash}
\end{tBox}

Once the application is finished, traces will be generated and stored in \texttt{`pwd`/traces} directory. 

Supported/Tested versions: Extrae 3.6.1 and COMPSs 2.4

\section{Federation with secure communications}

In this section we explain how to secure dataClay communications between different dataClay instances.

In a federated environment, different dataClays are communicating to each other via LogicModule service. 

Current implementation of dataClay does not allow you to specify certificates in server-side but only client certificates. 
Thus, we use a Traefik reverse-proxy \href {https://docs.traefik.io/} {https://docs.traefik.io/} 
to check client certificates and also to not publish dataClay ports. 

An example of the \textit{docker-compose.yml} file with reverse-proxy:


\begin{tBox}
 \begin{lstlisting}[language=docker-compose-2, frame=none]
version: '3.4'
services:

  proxy:
    image: traefik:v1.7.17
    restart: unless-stopped
    command: --docker --docker.exposedByDefault=false
    volumes:
      - /var/run/docker.sock:/var/run/docker.sock:ro
      - /home/docker/common/dataclay/traefik.toml:/traefik.toml
      - /home/docker/certs:/ssl:ro
    ports:
      - "80:80"
      - "443:443"
      
  logicmodule:
    image: "bscdataclay/logicmodule:2.0"
    environment:
      - LOGICMODULE_PORT_TCP=11034
      - LOGICMODULE_HOST=logicmodule
      - DATACLAY_ADMIN_USER=admin
      - DATACLAY_ADMIN_PASSWORD=admin
    volumes:
      - ./prop/global.properties:/usr/src/dataclay/javaclay/cfgfiles/global.properties:ro
      - ./prop/log4j2.xml:/usr/src/dataclay/javaclay/log4j2.xml:ro
    healthcheck:
       interval: 5s
       retries: 10
       test: ["CMD-SHELL", "/usr/src/dataclay/javaclay/health_check.sh"]
    labels:
      - "traefik.enable=true"
      - "traefik.backend=logicmodule"
      - "traefik.frontend.rule=Headers: service-alias,logicmodule"
      - "traefik.port=11034"
      - "traefik.protocol=h2c"
          
  dsjava:
    image: "bscdataclay/dsjava:2.0"
    depends_on:
      - logicmodule
    environment:
      - DATASERVICE_NAME=DS1
      - DATASERVICE_JAVA_PORT_TCP=2127
      - LOGICMODULE_PORT_TCP=11034
      - LOGICMODULE_HOST=logicmodule
    volumes:
      - ./prop/global.properties:/usr/src/dataclay/javaclay/cfgfiles/global.properties:ro
      - ./prop/log4j2.xml:/usr/src/dataclay/javaclay/log4j2.xml:ro
    healthcheck:
       interval: 5s
       retries: 10
       test: ["CMD-SHELL", "/usr/src/dataclay/javaclay/health_check.sh"]
       
  dspython:
    image: "bscdataclay/dspython:2.0"
    depends_on:
      - logicmodule
      - dsjava
    environment:
      - DATASERVICE_NAME=DS1
      - LOGICMODULE_PORT_TCP=11034
      - LOGICMODULE_HOST=logicmodule
      - DATASERVICE_PYTHON_PORT_TCP=6867
    volumes:
      - ./prop/global.properties:/usr/src/dataclay/pyclay/cfgfiles/global.properties:ro
    healthcheck:
       interval: 5s
       retries: 10
       test: ["CMD-SHELL", "/usr/src/dataclay/pyclay/health_check.sh"]
 \end{lstlisting}
\end{tBox}

With following \textit{traefik.toml} example:

\begin{tBox}
 \begin{lstlisting}[language=docker-compose-2, frame=none]
debug = false
defaultEntryPoints = ["http", "https"]
[entryPoints]
    [entryPoints.http]
    address = ":80"
        [entryPoints.http.redirect]
        entryPoint = "https"

   [entryPoints.https]
     address = ":443"
   [entryPoints.https.tls]
     [entryPoints.https.tls.clientCA]
     files = ["/ssl/dataclay-ca.crt"]
     optional = false

	 [entryPoints.https.tls.defaultCertificate]
     certFile = "/ssl/dataclay-agent.crt"
     keyFile  = "/ssl/dataclay-agent.pem"
     
     # For secure connection on frontend.local
     [[entryPoints.https.tls.certificates]]
     certFile = "/ssl/dataclay-agent.crt"
     keyFile  = "/ssl/dataclay-agent.pem"
 \end{lstlisting}
\end{tBox}

Note that ports are not published in \textit{docker-compose.yml} and we configure Traefik by adding labels to logicmodule service.

Finally, we need to configure our application to use the certificates. For that we have the following \textit{global.properties} options:

\begin{table}[H]
\footnotesize
\begin{tBox}
\centering
\begin{tabular}{p{57mm} | p{27mm} |  >{\raggedright\arraybackslash}p{50mm}}
\textbf{property} & \textbf{default value} & \textbf{description} \\
\hline
\verb LM_SERVICE_ALIAS_HEADERMSG & logicmodule & Add to the message the header service-alias (used to filter in traefik) . \\
\hline
\verb SSL_TARGET_AUTHORITY & proxy  & Override target authority (usually traefik service name). \\
\hline
\verb SSL_CLIENT_TRUSTED_CERTIFICATES & None & Path to CA certificate. \\
\hline
\verb SSL_CLIENT_CERTIFICATE & None & Path to Client certificate. \\
\hline
\verb SSL_CLIENT_KEY & None & Path to Client key. \\
\end{tabular}
\label{table:Security}
\end{tBox}
\end{table}

An example of the \textit{global.properties} file with TLS could be

\begin{tBox}
 \begin{bash}
LM_SERVICE_ALIAS_HEADERMSG=logicmodule
SSL_TARGET_AUTHORITY=proxy 
SSL_CLIENT_TRUSTED_CERTIFICATES=/usr/src/demo/app/certs/dataclay-ca.crt
SSL_CLIENT_CERTIFICATE=/usr/src/demo/app/certs/dataclay-agent.crt
SSL_CLIENT_KEY=/usr/src/demo/app/certs/dataclay-agent.pem
 \end{bash}
\end{tBox}