\chapterimage{Client.jpg} % Chapter heading image

\chapter{Client configuration}

\section{Client libraries}
\label{sec:ClientLibraries}

In order to connect your applications with dataClay services you need a client library for your preferred programming language.

If you are developing a Java application you can obtain the library following instructions in \ref{sec:FullDemo} (downloading our zip file as exposed in Section~\ref{sec:DownloadZIP} or following instructions for POM-based projects in Section~\ref{sec:POMbasedProjects}).

In case you are developing a Python application, you can easily install the Python module with \textit{pip} command:

\begin{tBox}
\begin{bash}
> pip install dataClay
\end{bash}
\end{tBox}

\section{Configuration files}
\label{sec:ClientConfigFiles}

The basic client configuration for an application is the minimum information required to initialize a session with dataClay. To this end two different files are required: the \textit{session.properties}\index{session.properties} file and the \textit{client.properties}\index{client.properties} file.

\subsection{Session properties}
This file contains the basic info to initialize a session with dataClay. It is automatically loaded during the initialization process (\texttt{DataClay.init()}\index{DataClay.init()} in Java or \texttt{api.init()}\index{api.init()} in Python) and its default path is \texttt{./cfgfiles/session.properties}. This path can be overridden by setting a different path through the environment variable \texttt{DATACLAYSESSIONCONFIG}. 

Here is an example:

\begin{tBox}
 \begin{bash}
  Account=MyAccount
  Password=MyPassword
  StubsClasspath=/home/me/myapp/stubs
  DataSetForStore=MyDataset
  DataSets=MyDataset,OtherDataSet
  LocalBackend=DS1
 % DataClayClientConfig=/home/me/myapp/client.properties
 \end{bash}
\end{tBox}

\texttt{Account} and \texttt{Password} properties are used to specify user's credentials. 

\texttt{StubsClasspath} defines a path where the stub classes can be located. That is, the path where \textit{dClayTool} (exposed in section \ref{sec:dClayTool}) saved our stub classes after calling \texttt{GetStubs} operation.

\texttt{DataSetForStore} specifies which dataset the application will use in case a \textit{makePersistent} request is produced to store a new object in the system, and \texttt{DataSets} provide information about the datasets the application will access (normally it includes the \texttt{DataSetForStore}). 

\texttt{LocalBackend} defines the default backend that the application will access when using either \texttt{DataClay.LOCAL} in Java or \texttt{api.LOCAL} in Python (examples of this can be found in API sections \ref{sec:JavaAPI} and \ref{sec:PythonAPI}). 

%Finally, the \texttt{DataClayClientConfig} contains a path pointing to \textit{client.properties} file, which is the second file required as exposed in next section.

\subsection{Client properties}
This file contains the minimum service info to connect applications with dataClay. It is also loaded automatically during the initialization process and its default path is \texttt{./cfgfiles/client.properties}, which can be overriden by setting the environment variable \texttt{DATACLAYCLIENTCONFIG}.

Here is an example:

\begin{tBox}
 \begin{bash}
 HOST=localhost
 TCPPORT=11034
 \end{bash}
\end{tBox}

As you can see, it only requires two properties to be defined: \texttt{HOST} and \texttt{TCPPORT}; comprising the full address to be resolved in order to initialize a session with dataClay from your application.
